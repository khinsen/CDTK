\documentclass[12pt]{article}
\usepackage{a4}

\newcommand{\vect}[1]{{\bf #1}}
\newcommand{\mat}[1]{{\bf #1}}

\newcommand{\va}{\vect{a}}
\newcommand{\vb}{\vect{b}}
\newcommand{\vq}{\vect{q}}
\newcommand{\vr}{\vect{r}}
\newcommand{\vs}{\vect{s}}
\newcommand{\vt}{\vect{t}}

\newcommand{\mU}{\mat{U}}

\begin{document}

{\parindent=0mm

{\Large
A Physicist's Guide to Crystallographic Data
}

\vspace{5mm}

{\large Konrad Hinsen$^{a, b}$}\\
\\
$^a$Centre de Biophysique Mol\'eculaire, CNRS UPR 4301
\footnote{Affiliated with the University of Orl\'eans}\\
Rue Charles Sadron\\
45071 Orl\'eans Cedex 2\\
France\\
\\
$^b$Synchrotron Soleil, Saint Aubin, B.P. 48\\
91192 Gif sur Yvette Cedex\\
France\\
\\
E-mail: hinsen@cnrs-orleans.fr\\
}
\date{}

\vspace{15mm}

\begin{abstract}
  This tutorial explains as much of crystallographers' jargon and
  conventions as is required to interpret crystallographic data
  such as the structure factor files published in the Protein
  Data Bank (PDB). It also explains how such data is represented
  in the Crystallographic Data Toolkit (CDTK), a Python library for
  working with crystallographic data.
\end{abstract}

\newpage
\begin{sloppy}

\section{Units}

Crystallographers measure lengths in {\AA}ngstr{\"om}s (\AA) and angles
in degrees. CDTK uses the same ``atomic SI'' unit system as the
Molecular Modelling Toolkit (MMTK), meaning that distances are
measured in nm and angles in radians.

The module \texttt{CDTK.Units} contains a large collection of unit
conversion factors from and to CDTK's internal units. It is a good
habit to use these factors in Python scripts to indicate the unit
of all input quantities. For example:

\begin{verbatim}
from CDTK import Units

length_1 = 2.5*Units.Ang  # Angstrom
length_2 = 0.5*Units.nm   # nanometer
big_length = 300.*Units.m # meter

angle_1 = 90.*Units.deg   # degrees
angle_2 = 1.3*Units.rad   # radians
\end{verbatim}

The same factors can be used for printing results in the desired units:

\begin{verbatim}
from CDTK import Units

print "the length is %f Angstrom" % (length/Units.Ang)
\end{verbatim}


\section{Crystals}

\subsection{The unit cell}

An ideal crystal is an arrangement of molecules that is periodic in
the three dimensions of space. Its basic motif is the \textit{unit cell},
whose most general shape is a parallelepiped defined by the three
\textit{lattice vectors} $\va_i$, $i=1, 2, 3$.
In PDB files, the shape of the unit cell is described
in terms of the six parameters
\begin{eqnarray}
a &=& |\va_1| \nonumber \\
b &=& |\va_2| \nonumber \\
c &=& |\va_3| \nonumber \\
\alpha &=& \arccos \va_2\cdot\va_3 \nonumber \\
\beta  &=& \arccos \va_1\cdot\va_3 \nonumber \\
\gamma &=& \arccos \va_1\cdot\va_2 \nonumber
\end{eqnarray}
that are given in the \texttt{CRYST1} record.
These six parameters define the shape of the unit cell, but not its
orientation in space. The orientation is defined by the convention
that $\va_1$ is parallel to the $x$-axis and that $\va_2$
lies in the $x$-$y$-plane.

In CDTK, the unit cell is defined by the class
\texttt{CDTK.Crystal.UnitCell}. The class constructor can either be
called with three vector arguments, defining the lattice vectors, or
with six number arguments that correspond to $a$, $b$, $c$, $\alpha$,
$\beta$, and $\gamma$. All CDTK routines that expect a unit cell
argument will also accept an MMTK universe object instead.

\subsection{Fractional and Cartesian coordinates}

Points in the unit cell can of course be identified by their Cartesian
coordinates in space. However, it is often practical to use
\textit{fractional coordinates} instead. Fractional coordinates
describe a point by displacements from the origin along the lattice
vectors. A point defined by the fractional coordinates $(x_1, x_2, x_3)$
has Cartesian coordinates $\vect{r} = \sum_{i=1}^3 x_i\va_i$.
For points inside the unit cell, the fractional coordinates are in the
range $0 <= x_i < 1$.

Conversion between fractional and Cartesian coordinates is provided by
CDTK's \texttt{UnitCell} class (and also by MMTK's universe classes).
The methods \texttt{fractionalToCartesian} and
\texttt{cartesianToFractional} convert one point, whereas the methods
\texttt{fractionalToCartesianMatrix} and
\texttt{cartesianToFractionalMatrix} return the $3 \times 3$
conversion matrix.

\subsection{Symmetry}

Symmetry operations are geometrical operations (translations,
rotations, reflections, inversions, and combinations thereof) that
leave a given crystal unchanged. All crystals share three basic
symmetry operations: translations along the three lattice vectors
$\va_i$.

Most crystals have additional symmetry operations that are defined by
its \textit{space group}. There are in total 230 space groups, but
only 65 of them can occur in crystals of biological macromolecules
because reflections and inversions are not admitted due to the
chirality of the molecules. In a PDB file, the space group of the
crystal is indicated in the \texttt{CRYST1} record by its name (e.g.
\texttt{P 21 21 21} or \texttt{C 2}). The set of symmetry operations
for a given space group can be looked up in tables. CDTK provides a
table containing all 230 space groups. This table is accessible as
\texttt{CDTK.SpaceGroups.space\_groups} and takes the form of a
dictionary that maps space group names or space group numbers (unique
numbers have been assigned by convention) to the corresponding space
group object.

A space group object stores the list of symmetry operations. Each
symmetry operation is represented by a rotation matrix $\mat{D}$ and a
translation vector $\vt$ and specifies that the point $\vr' =
\mat{D}\cdot \vr + \vt$ is equivalent to point $\vr$. The first
element of the symmetry operation list is always the identity
operation ($\mat{D}=\mat{1}$, $\vt=\vect{0}$).

The symmetry operations defined by the space group reduce the amount
of information required to describe the contents of the unit cell. If
the space group has $N$ symmetry operations, the number of independent
atom specifications is reduced by a factor of $N$. A subset of the
unit cell from which all the other atoms can be reconstructed using
symmetry operations is called an \textit{asymmetric unit}. This subset
is of course not unique. The atom records in a PDB file describe an
asymmetric unit.


\section{Structure factors}

The central quantity in the description of X-ray scattering from
crystals is the \textit{structure factor} defined by
\begin{equation}
\label{eq:sf}
F(\vs) = \sum_k f_k(|\vs|) e^{2\pi i\vs\cdot\vr_k},
\end{equation}
where $\vr_k$ is the position of atom $k$ and $f_k(|\vs|)$ is its
\textit{atomic scattering factor}. The summation is performed over all
atoms in the crystal.

The atomic scattering factor describes the electron cloud around an
atom and depends on its chemical element and its ionization state. For
numerical calculations, it is most commonly approximated by a sum of
four Gaussians plus a constant:
\begin{equation}
f(s) = \sum_{i=1}^5 a_i e^{-b_i s^2}
\end{equation}
with $b_5 = 0$. The parameters of this approximation have been tabulated
\cite{}. In CDTK, this table is available in the form of the
dictionary
\texttt{CDTK.AtomicScatteringFactors.atomic\_scattering\_factors}.

Since the atomic scattering factor is the Fourier transform of the
electron density of a single atom, the structure factor defined
by Eq.~(\ref{eq:sf}) is the Fourier transform of the electron density
of the whole crystal. Note that physicists would write it in terms
of the vector $\vq = 2\pi\vs$.


\subsection{Bragg reflections}

For an ideal crystal, the structure factor defined by
Eq.~(\ref{eq:sf}), being the Fourier transform of a periodic function,
is non-zero only at discrete points in reciprocal space. These points
are known as \textit{Bragg reflections} (usually just called
``reflections''). They lie on the \textit{reciprocal lattice} and are
given by
\begin{equation}
\vs_{hkl} = h \vb_1 + k \vb_2 + l \vb_3
\end{equation}
where $h, k, l$ are integer numbers called the \textit{Miller indices}
of a reflection and $\vb_i$ are the reciprocal basis vectors defined by
\begin{equation}
\va_i \cdot \vb_j = \delta_{ij}.
\end{equation}
Moreover, the summation in Eq.~(\ref{eq:sf}) can be restricted to the
atoms in the unit cell for an ideal crystal.

The quantity measured in a standard crystallographic experiment is the
square of the magnitude of the structure factor at the reciprocal
lattice points, i.e.
\begin{equation}
I(\vs_{hkl}) = \left| F(\vs_{hkl}) \right|^2,
\end{equation}
which is called the \textit{intensity} of the reflection $\vs_{hkl}$.

The quantity $d_{hkl}=1/|\vs_{hkl}|$ is called the \textit{resolution}
of a reflection. When a crystal structure is described as ``at $2 \AA$
resolution'', this means that there were observable reflections up to
$|\vs|=0.5 \AA^{-1}$.
\begin{quote}
\textbf{Watch out: ``high resolution'' means large $|\vs|$ and thus small $d$;
a resolution of $1 \AA$ is higher than a resolution of $2 \AA$!}
\end{quote}

In practice there is also a lower limit to the resolution of the
available reflections, meaning that the result of a crystallographic
experiment consists of the intensities $I(\vs_{hkl})$ for the
reflections in the range $1/d_{\mbox{low}} \leq \vs_{hkl} \leq
1/d_{\mbox{high}}$. $d_{\mbox{low}}$ and $d_{\mbox{high}}$ are
specified as the \textit{resolution range} in a PDB file (in the
\texttt{REMARK   3} section).

The mmCIF structure factor files available for many structures from
the PDB contain $I(\vs_{hkl})$ (mmCIF label \texttt{intensity\_meas})
or $\sqrt{I(\vs_{hkl})}$ (mmCIF label \texttt{F\_meas} or
\texttt{F\_meas\_au}) for the reflections in the specified resolution
range. However, not all reflections are given explicitly, because the
number of independent reflections is reduced by symmetry:
\begin{enumerate}
\item
In the absence of anomalous scattering, the atomic scattering factors
$f_k(|\vs|)$ are real and the structure factor has the symmetry
$F(-\vs) = F^{*}(\vs)$.
\item
The symmetry operations of the space group also apply in reciprocal
space. A symmetry relation defined by ($\mat{D}$, $\vt$) in real space
implies that the reciprocal space point $\vs' = \mat{D}^T\cdot\vs$
is equivalent to $\vs$.
\item
If the space group has symmetry operations with non-zero translations,
the intensities of certain reflections must vanish. These reflections
are called \textit{systematic absences}.
\end{enumerate}
A structure factor file usually lists only a minimal set of
independent reflections. However, the choice of these reflections is
not unique.

\vspace{3mm}

In CDTK, the class \texttt{Reflection} represents a single reflection
defined by its Miller indices. The class \texttt{ReflectionSet}
represents the spherical shell in reciprocal space that corresponds to
a particular crystallographic experiment. A \texttt{ReflectionSet} stores
only a minimal set of reflections explicitly, and it provides iteration
over this minimal set. However, it can also provide a \texttt{Reflection}
object for any set of Miller indices inside the spherical shell.

Data defined for each reflection (intensities, structure factors,
structure factor amplitudes) are not stored in a
\texttt{ReflectionSet}, but in separate classes
(\texttt{ExperimentalIntensities}, \texttt{ExperimentalAmplitudes},
\texttt{StructureFactor}, \texttt{ModelAmplitudes},
\texttt{ModelIntensities}) that store a reference to a
\texttt{ReflectionSet}. The reason for this is that there are usually
several data sets for a single \texttt{ReflectionSet} object. In a
typical workflow, parsing a reflection file yields a
\texttt{ReflectionSet} and an associated
\texttt{ExperimentalIntensities} or \texttt{ExperimentalAmplitudes}
object. Then a \texttt{StructureFactor} object is generated from
a model and compared to the experimental data.

\subsection{Crystallographic models}

Real biomolecular crystals are not perfectly periodic. The arrangement
of the molecules in the different copies of the unit cell varies due
to crystal defects and thermal motion. As a consequence, the structure
factor in Eq.~(\ref{eq:sf}) is non-zero for points in reciprocal space
outside the reciprocal lattice; this is known as \textit{diffuse
scattering}. The energy that goes into diffuse scattering is also
lost to the Bragg reflections, whose intensity is reduced by
destructive interference between the contributions from unit cell
copies with slightly different conformations. This attenuation of the
Bragg peaks becomes more important with increasing $|\vs|$.

Crystallographic models represent non-ideal crystals by an average
conformation plus a fluctuation around it. Eq.~(\ref{eq:sf}) is
replaced by
\begin{equation}
\label{eq:model_sf}
F(\vs) = \sum_k f_k(|\vs|) e^{-2\pi^2 \vs\cdot\mU_k\cdot\vs} e^{2\pi i\vs\cdot<\vr_k>},
\end{equation}
with a summation over the atoms in the unit cell.
The additional factor $\exp(-2\pi^2 \vs\cdot\mU_k\cdot\vs)$,
called the \textit{Debye-Waller factor}, describes the fluctuation
of an atom in an harmonic potential well. The symmetric tensor $\mU_k$
describes the position fluctuations of atom~$k$; it is given by
\begin{equation}
\mU_k = \left<  (\vr_k-<\vr_k>) (\vr_k-<\vr_k>) \right>.
\end{equation}
Its elements are called \textit{anisotropic displacement parameters}
(ADPs) by crystallographers. Their use in biomolecular crystallography
is relatively recent, because high-resolution data is required to fit
nine parameters per atom (three for the position, six for the
fluctuations). For lower-resolution experiments, the fluctuation tensor
is assumed to be isotropic, reducing it to a single parameter known
as the \textit{B factor} and given by
\begin{equation}
B_k = \frac{8\pi^2}{3} \mbox{tr } \mU_k.
\end{equation}

Two points are important to note:
\begin{enumerate}
\item
  Eq.~(\ref{eq:model_sf}) is an approximation. The conformational
  variability of the crystal need not be described by Gaussian
  fluctuations.
\item
  The Debye-Waller factor was originally derived as a model for
  thermal fluctuations in an harmonic potential, and the B~factor is
  often called ``temperature factor'' for this reason. However, the
  conformational variability it is made to describe in practice is
  only partly due to thermal fluctuations.
\end{enumerate}

A complementaary approach to modelling conformational variability is
the use of multiple position/fluctuation sets for some atoms. In this
case, each set carries a weight called the \textit{occupancy}; the sum
of the occupancies over all position/fluctuation sets for one atom
must be~1.

\vspace{3mm}

CDTK permits the calculation of structure factors from standard
crystallographic models for a given \texttt{ReflectionSet}.
There are three ways to specify the model:
\begin{itemize}
\item
as an iterator over the atoms in the unit cell, yielding a tuple
of (chemical element, position, fluctuation tensor, occupancy)
for each atom:
{\small
\begin{verbatim}
sf = StructureFactor(reflection_set)
sf.calculateFromUnitCellAtoms((atom.symbol, atom.position(),
                               atom.temperature_factor/(8.*N.pi**2),
                               atom.occupancy)
                              for atom in unit_cell.atomList())
\end{verbatim}
}
\item
as an iterator over the atoms in the asymmetric unit, yielding a tuple
of (chemical element, position, fluctuation tensor, occupancy)
for each atom:
{\small
\begin{verbatim}
sf = StructureFactor(reflection_set)
sf.calculateFromAsymmetricUnitAtoms((atom.symbol, atom.position(),
                                     adps[atom], atom.occupancy)
                                    for atom in asu.atomList())
\end{verbatim}
}
\item
as an MMTK universe plus an optional MMTK \texttt{ParticleTensor} object
specifying the fluctuation tensors:
{\small
\begin{verbatim}
sf = StructureFactor(reflection_set)
sf.calculateFromUniverse(universe, adps)
\end{verbatim}
}
\end{itemize}

\subsection{Comparing models to experimental data}

The most commonly used similarity criterion for comparing a model
to an experimental data set is the R~factor, defined by
\begin{equation}
R = \frac{\sum_{hkl}
    \left| |F_{\mbox{model}}(\vs_{hkl})|-|F_{\mbox{experiment}}(\vs_{hkl})| \right|}
    {\sum_{hkl} |F_{\mbox{experiment}}(\vs_{hkl})|}.
\end{equation}
The summation is performed over the minimal set of unique reflections.
The experimental structure factor amplitudes are simply the square roots
of the measured intensities.

In CDTK, the R~factor is calculated by
\begin{verbatim}
r = exp_amplitudes.rFactor(model_sf)
\end{verbatim}
Since experimental intensities are usually not normalized, it is often
preferable to allow a scale factor between the two data sets that
minimizes the R~factor. This is written as
\begin{verbatim}
r, scale = exp_amplitudes.rFactorWithScale(model_sf)
\end{verbatim}

%\begin{equation}
%\end{equation}

\end{sloppy}
\end{document}
